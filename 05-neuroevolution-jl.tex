\chapter{Developping NeuroEvolution.jl}
\label{sec:neuroevo}

\hrefhttps://github.com/TemplierPaul/NeuroEvolution.jl{}{\color{blue}{NeuroEvolution.jl}} is a Julia library I developed to implement multiple neuroevolution algorithms

\section{NEAT implementation}

I developed the NEAT algorithm based on the original papers \cite{NEAT_1} and \cite{NEAT_2}, taking into account the remarks and indications of the author on \href{https://www.cs.ucf.edu/~kstanley/neat.html#FAQ2}{
\color{blue}{his website}}. \\
The \href{https://github.com/FernandoTorres/NEAT}{\color{blue}{reorganized version}} of the original C++ implementation of NEAT also provided insight into the algorithm, although some parts were approached differently, taking the papers as reference.   

\subsection{NEAT Individual}

\subsection{Neuron position and Execution order}
This implementation of NEAT was inspired by the work of \cite{wilson2018positionalcgp} in which the position of a CGP node is described as a floating point value. This representation is here used to determine the order of computation of neuron output values, hence defining the architecture.  

The positions are defined as follows:
\begin{itemize}
    \item All input neurons are placed with negative integer positions.  
    \item The bias neuron (constant equal to 1) is placed at $0$.  
    \item All output neurons are placed at positive integer positions, starting at $1$.
    \item All other neurons are placed with positions in $]0, 1[$.
\end{itemize}

Neurons added through mutation of a connection are placed at a random position between the origin node of the connection at $p_o$ and the destination node at $p_d$, so that the order of execution is not modified. \\
The position of the new neuron is therefore randomly drawn in $]p_o, p_d[ \: \cap \: ]0, 1[$.




\section{Adding HyperNEAT}

\section{CMA-ES applied to neuroevolution}

%%% Local Variables: 
%%% mode: latex
%%% TeX-master: "isae-report-template"
%%% End: 