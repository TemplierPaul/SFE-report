\chapter{Prolegomena}
\label{chap:prolegomena}

\section{Reinforcement Learning: what is an MDP?}
      
\begin{figure}
    \centering
    \begin{tikzpicture}

  \node[draw] (A) at (0cm,0cm) {State s};
  \node[draw] (B) at (3cm,1.5cm) {State s'};
  \node[draw] (C) at (3cm,-1.5cm) {State s''};
  
  \node[draw,fill=yellow] (Arga) at (1.5cm,2.5cm) {Action a'};
  \node[draw,fill=yellow] (Argb) at (1.5cm,-2.5cm) {Action a''};
  
  \draw node[vertex] (Jointa) at (1.5cm,0.75cm) {};
  \draw node[vertex] (Jointb) at (1.5cm, -0.75cm) {};
  
  \draw[->,draw=blue] (A) to (B);
  \draw[->,draw=blue] (A) to (C);
  
  \draw[-,draw=black] (Arga) to (Jointa);
  \draw[-,draw=black] (Argb) to (Jointb);
  
  \end{tikzpicture}
  
    \caption{Markov Decision Process}
    \label{fig:my_label}
\end{figure}


\section{Neural Networks acting as a policy}

\section{Gradient-based optimization}

\section{Evolutionary Reinforcement Learning}
ERL is RL with evolutionary methods

The ERL field lacks a comparison / benchmarking tool, which is why we did the DOTA 2 competition and we are working on BERL

\section{Development tools}
\subsection{Julia}
    
\addlink{https://julialang.org/}{\textbf{Julia}} is a high-level, dynamic programming language designed for high-performance with a just-in-time compiler. It is notably interoperable with multiple languages, including C, Fortran, Python, R, MATLAB, Java, or Scala. \cite{julia-lang}

We used Julia in this project to provide a faster environment than Python, while still being compatible with state-of-the-art Python libraries such as gym or Atari.
\\

The principles of \textbf{Test-Driven Development} were followed: requirements are first translated into tests, then the code is developed until all tests pass. Tests provide both a way to ensure new code doesn't break previously working code, and a list of use cases of the library, hence improving its documentation.

Additionally a \addlink{https://travis-ci.org/}{\textbf{TravisCI}} instance was linked to all libraries I developed to build and test all code on remote servers, ensuring they can be run on machines with a clean configuration. 

\subsection{Cambrian}

\addlink{https://github.com/d9w/Cambrian.jl}{\textbf{Cambrian.jl}} is an Evolutionary Computation framework developed in Julia by my supervisor Dennis Wilson. It implements a structure to facilitate the development of evolutionary algorithms, focusing on genetic programming and neuroevolution. 

I used Cambrian in to provide a consistent framework between projects and ensure compatibility, and for its usefulness in structuring the development.

%%% Local Variables: 
%%% mode: latex
%%% TeX-master: "isae-report-template"
%%% End: 